%---------------------------------------------------------
\cvsection{Projects}
%---------------------------------------------------------
\cvprojectssubtitle{Full list of repositories available on GitHub}

\begin{cvprojects}
  %---------------------------------------------------------
  \cvproject{hoppingmode-web-frontend}
  {Personal portfolio website showcasing my GitHub repositories.}
  {
    \begin{itemize}
      \item
            \cvprojectbullet{Languages}{TypeScript, nodejs, GraphQL}
      \item
            \cvprojectbullet{Technologies}{nodejs, Docker, docker-compose, nginx, GitHub Actions, ESLint, Prettier, Husky}
      \item
            \cvprojectbullet{Motivation}{
              \begin{cvprojectbulletitems}
                \item Learn new web frameworks and technologies.
                \item Enhance the scope of my full-stack engineering skills
                \item Create a platform to showcase my existing work in a visually appealing way.
              \end{cvprojectbulletitems}
            }
    \end{itemize}
  }

  %---------------------------------------------------------
  \cvproject{hoppingmode-web-api}
  {RESTful API that acts as a proxy to the GitHub API to provide the portfolio web application convenient access to my GitHub repositories}
  {
    \begin{itemize}
      \item
            \cvprojectbullet{Languages}{TypeScript, nodejs, GraphQL}
      \item
            \cvprojectbullet{Technologies}{Express, nodejs, Docker, docker-compose, nginx, GitHub Actions, ESLint, Prettier, Husky}
      \item
            \cvprojectbullet{Motivation}{
              \begin{cvprojectbulletitems}
                \item Create an astraction over GitHub's REST and GraphQL APIs to query my public repositories.
                \item Transform the data returned by the GitHub API into a format that is more convenient for my portfolio website to consume.
                \item Deal with API authentication and key management.
              \end{cvprojectbulletitems}
            }
    \end{itemize}
  }

  %---------------------------------------------------------
  \cvproject{SmellSense}
  {An iOS/Android application that facilitates smell training for those suffering from loss of smell and taste}
  {
    \begin{itemize}
      \item
            \cvprojectbullet{Languages}{Dart}
      \item
            \cvprojectbullet{Technologies}{Flutter, HiveDB}
      \item
            \cvprojectbullet{Motivation}{
              \begin{cvprojectbulletitems}
                \item
                Inspired by the COVID-19 pandemic and my father's desire as an Ear, Nose and Throat surgeon to help his
                patients suffering from the secondary, more long-term symptoms of the virus.
                \item Help improve the quality of life of those suffering from loss of smell and taste
              \end{cvprojectbulletitems}
            }
      \item \cvprojectbullet{Features}{
              \begin{cvprojectbulletitems}
                \item Timed smell training sessions
                \item Log entries during training
                \item View training history, ratings, and improvement over time
              \end{cvprojectbulletitems}
            }
      \item
            \cvprojectbullet{Achievement}{Deployed to the App Store and Google Play Store}
    \end{itemize}
  }

  %---------------------------------------------------------
  \cvproject{Desktop Toolkit (ongoing)}
  {Desktop application containing a collection of tools for everyday use}
  {
    \begin{itemize}
      \item
            \cvprojectbullet{Languages}{TypeScript}
      \item
            \cvprojectbullet{Technologies}{Electron, Redux, nodejs}
      \item
            \cvprojectbullet{Motivation}{
              \begin{cvprojectbulletitems}
                \item Learn React and Electron by building a desktop application
                \item Remove dependence on external third-party tools
              \end{cvprojectbulletitems}
            }
    \end{itemize}
  }

  %---------------------------------------------------------
  \cvproject{vscode-code-repositories-extension}
  {Manage your local code repositories directly through Visual Studio Code.}
  {
    \begin{itemize}
      \item
            \cvprojectbullet{Languages}{TypeScript}
      \item
            \cvprojectbullet{Technologies}{Visual Studio Code, nodejs}
    \end{itemize}
  }

\end{cvprojects}